\documentclass{article}
\usepackage[utf8]{inputenc}
\usepackage{bidi}
\usepackage{geometry}
\geometry{margin=2.5cm}
\begin{document}
\section*{טבלת ציונים}
\begin{RTL}
\begin{tabular}{|c|c|c|p{10cm}|}
\hline
שאלה & סעיף & ציון & הערה \\
\hline
1 & א & 90 & יפה, אבל אפשר היה להציג את N בצורה יותר אלגנטית.  ההסבר קצת ארוך מדי. \\ \hline
1 & ב & 95 & פתרון מעולה! נימוק מדויק וברור. \\ \hline
2 & א & 75 & ההוכחה לא ברורה מספיק.  חסר ניסוח מדויק של השלבים. \\ \hline
2 & ב & 70 & הלוגיקה לא ברורה לחלוטין.  יש לשפר את ההסבר. \\ \hline
3 & א & 85 & ההוכחה נכונה, אבל ניסוח לא מדויק.  יש לשפר את הצגת השלבים. \\ \hline
3 & ב & 92 & פתרון מצוין! הוכחה ברורה ומדויקת. \\ \hline
3 & ג & 80 & ההוכחה טובה, אבל ניתן היה לפשט אותה.  הצגת הטיעון אינה מספיק חלקה. \\ \hline
3 & ד & 78 & הוכחה נכונה, אך חסר פירוט מספיק.  הייתי מצפה להסבר מפורט יותר. \\ \hline
3 & ה & 65 & הוכחת ההתבדרות לא מספקת.  יש להציג דוגמה נגדית ברורה יותר. \\ \hline
3 & ו & 88 & הוכחה טובה, נימוק ברור.  יפה! \\ \hline
4 & א & 100 & הוכחה אלגנטית ומדויקת! \\ \hline
4 & ב & 0 & דוגמא נגדית לא נכונה.  יש להציג דוגמא נגדית תקפה. \\ \hline
4 & ג & 100 & הוכחה מעולה!  שימוש נכון בסעיף הקודם. \\ \hline
5 & א & 90 & הנימוק קצר מדי.  יש להרחיב ולפרט יותר. \\ \hline
5 & ב & 90 & הנימוק קצר מדי.  יש להרחיב ולפרט יותר. \\ \hline
5 & ג & 70 & הדוגמא נגדית לא מספיקה.  יש להוסיף הסבר מפורט יותר. \\ \hline
5 & ד & 95 & נימוק מצוין! \\ \hline
5 & ה & 60 & הנימוק לא ברור מספיק.  יש לשפר את ההסבר. \\ \hline
5 & ו & 75 & הדוגמא נגדית לא מספיקה.  יש להוסיף הסבר מפורט יותר. \\ \hline
5 & ז & 100 & נימוק מצוין! \\ \hline
\end{tabular}
\end{RTL}
\end{document}