\documentclass{article}
\usepackage[utf8]{inputenc}
\usepackage{bidi}
\usepackage{geometry}
\geometry{margin=2.5cm}
\begin{document}
\section*{טבלת ציונים}
\begin{RTL}
\begin{tabular}{|c|c|c|p{10cm}|}
\hline
שאלה & סעיף & ציון & הערה \\
\hline
1 & 1.1 & 90 & הפתרון נכון מבחינה רעיונית.  השימוש ב- N=2022 נכון, ומתקיים d(a<sub>n</sub>, 1) ≤ 0.001 עבור כל n > N. עם זאת,  הניסוח אינו מדויק.  הטענה 'ונבחרת 1 גבול הסדרה הוא' אינה מתאימה.  מומלץ לנסח בצורה מדויקת יותר את  הקשר בין  N לבין  ε  ולא להסתפק באומדן.  חסר הסבר מדוע  d(a<sub>n</sub>, 1) = 0  עבור n > 2021. \\ \hline
1 & 1.2 & 80 & הפתרון נכון מבחינה רעיונית, ומציג את הגבול הנכון.  עם זאת, הנימוק אינו מספיק משכנע.  הטענה 'n≥201 ונסיק שהפסוק נכון עבור כל n=201 גדל' אינה ברורה ומנומקת בצורה לא מספקת.  הצבת n=201 אינה הוכחה מספקת שזה נכון לכל n>200.  היה צריך להראות  באופן כללי כיצד  |a<sub>n</sub> - 5/6| קטן מ- 0.001  עבור n גדול מספיק. \\ \hline
2 & 2.1 & 70 & ההוכחה אינה מלאה.  הטענה 'm+n > n > N וממילא m+n > n, אז m > 0' אינה רלוונטית.  ההוכחה צריכה להראות שניתן למצוא N  מתאים כך שלכל n > N,  |a<sub>m+n</sub> - L| < ε.  ההוכחה לא מראה באופן ברור את הקשר בין N  עבור  (a<sub>n</sub>) לבין N  עבור  (b<sub>n</sub>). \\ \hline
2 & 2.2 & 60 & הפתרון אינו ברור ומסורבל.  ההוכחה אינה מובנת היטב.  הקשר בין N<sub>0</sub> ו- N<sub>1</sub> אינו ברור מספיק.  הניסוח  'ai מתכוון לכל a<sub>n+m</sub>' אינו מדויק.   ההוכחה צריכה להראות בבירור כיצד ניתן לבחור N<sub>1</sub>  בהתבסס על  N<sub>0</sub>  כך שתתקיים הדרישה  |a<sub>n</sub> - L| < ε  עבור כל n > N<sub>1</sub>. \\ \hline
3 & 3.1 & 50 & הפתרון מכיל טעויות רבות.  המעבר מ- (3n<sup>5</sup> + 1)/(4n<sup>6</sup> + n) ל- (3n<sup>5</sup> + 1)/n<sup>6</sup>  אינו נכון.  הגדלת המונה והקטנת המכנה משנים את אי השוויון.   המעבר ל- 4/n  אינו ברור.  ההוכחה אינה שלמה ואינה מראה כיצד לבחור N  בהתאם ל- ε. \\ \hline
3 & 3.2 & 40 & הפתרון מכיל טעויות לוגיות וניסוח לא מדויק.  המעבר מ- |(5 + (-1)<sup>n</sup>n)/(4n<sup>2</sup> + 1)|  ל- |n(1 + (-1)<sup>n</sup>)/(4n<sup>2</sup> + 1)| אינו תקף באופן כללי.  הטיעון לגבי n זוגי ואי זוגי אינו מנומק היטב, וההוכחה לא שלמה.  לא הוכח כיצד לבחור N  בהתאם ל- ε. \\ \hline
3 & 3.3 & 30 & הפתרון מכיל שגיאות חישוב וטעויות לוגיות.  ההצדקה להשלמת הריבוע אינה ברורה.  המעבר מ- √(16n<sup>2</sup> + 3) / n - 4  ל-  √(16n<sup>2</sup> + 3 - 4n)/n אינו נכון.  האי-שוויון  √(16n<sup>2</sup> + 3 - 4n) < 3  אינו תקף.  ההוכחה אינה שלמה ואינה מראה כיצד לבחור N בהתאם ל- ε. \\ \hline
3 & 3.4 & 70 & הפתרון נכון מבחינה רעיונית.  המעבר מ- √(n<sup>2</sup> + 1) - n  ל- 1/(√(n<sup>2</sup> + 1) + n)  נכון.  האי-שוויון 1/(√(n<sup>2</sup> + 1) + n) < 1/n  נכון.  עם זאת,  חסר הסבר מדויק כיצד לבחור N בהתאם ל- ε.  הניסוח יכול להיות מדויק יותר. \\ \hline
3 & 3.5 & 85 & הפתרון נכון מבחינה רעיונית.  ההוכחה  מתבססת נכון על ההגדרה של התבדרות, ומראה כיצד למצוא  ε  מתאים לכל  L.   עם זאת, הניסוח  יכול להיות ברור יותר,  ומומלץ להסביר בצורה מפורטת יותר את בחירת  n  בהתאם ל- N ו-  ε. \\ \hline
3 & 3.6 & 65 & הפתרון מראה הבנה של מושג ההתבדרות.  הדוגמאות של L=1 ו- L=-1  ממחישות את ההתבדרות.  עם זאת, ההסבר אינו מספיק פורמלי.  היה צריך להוכיח בצורה  מדויקת יותר שקיים ε  כך שלכל N קיים n > N שמקיים |a<sub>n</sub> - L| ≥ ε  לכל  L. \\ \hline
4 & 4.1 & 95 & ההוכחה נכונה ושלמה.  הוכח  בצורה מדויקת את שוויון הגבולות בשני הכיוונים.  הניסוח ברור ומובן. \\ \hline
4 & 4.2 & 100 & הדוגמא הנגדית מוצגת בצורה נכונה ומדויקת, ומפריכה את הטענה.  ההסבר ברור ומובן. \\ \hline
4 & 4.3 & 90 & הפתרון נכון.  ההסתמכות על סעיף 4.1  נכונה, אך ניתן היה להוסיף  הסבר קצר יותר כיצד סעיף 4.1  מוביל למסקנה  בסעיף 4.3. \\ \hline
5 & 5.1 & 100 & הנימוק קצר, מדויק ונכון. \\ \hline
5 & 5.2 & 95 & הנימוק טוב,  אבל ניתן היה לנסח בצורה  קצת יותר  קומפקטית ומדויקת. \\ \hline
5 & 5.3 & 80 & הדוגמה  ממחישה את הטענה,  אבל  ההסבר  יכול להיות יותר  מדויק.  חסר הסבר  למה הדוגמה  אינה מתאימה להגדרה המקורית. \\ \hline
5 & 5.4 & 100 & הנימוק קצר, מדויק ונכון. \\ \hline
5 & 5.5 & 75 & הנימוק  טוב,  אבל  הוא לא  מפורט מספיק.  היה  צריך להראות  באופן  ברור  למה  הדוגמה  מפריכה  את  הטענה. \\ \hline
5 & 5.6 & 100 & הדוגמה הנגדית מוצגת בצורה נכונה ומדויקת, ומפריכה את הטענה.  ההסבר ברור ומובן. \\ \hline
5 & 5.7 & 100 & הנימוק קצר, מדויק ונכון. \\ \hline
\end{tabular}
\end{RTL}
\end{document}