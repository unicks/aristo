\documentclass{article}
\usepackage[utf8]{inputenc}
\usepackage{bidi}
\usepackage{geometry}
\geometry{margin=2.5cm}
\begin{document}
\section*{טבלת ציונים}
\begin{RTL}
\begin{tabular}{|c|c|c|p{10cm}|}
\hline
שאלה & סעיף & ציון & הערה \\
\hline
1 & א & 70 & הפתרון מראה הבנה בסיסית של מושג הגבול, אך הנימוק אינו מספיק מדויק.  הטענה ש-an = 1 עבור כל n > N אינה נכונה באופן כללי.  הפתרון צריך להראות כיצד נבחר N עבור ε נתון, ולהוכיח ש עבור כל n>N מתקיים |an - L| < ε.  השימוש בדוגמאות ספציפיות (N=2021, N=444) אינו מספיק להוכחת הטענה באופן כללי.  חסר הסבר מפורש כיצד הגיע למסקנה שגבול הסדרה הוא 5/6. \\ \hline
1 & ב & 30 & הפתרון לא ברור ומבולגן.  הניסוח לא מדויק, וקשה לעקוב אחר השלבים.  השימוש בסימון (♡) אינו ברור ואינו תורם להבנה.  אין הוכחה מסודרת של בחירת N  עבור ε נתון.  הטענה ש-4/(9n-3) < 4 אינה ברורה, ויש להסביר אותה בצורה מתמטית מדויקת. \\ \hline
2 & א & 90 & הפתרון נכון מבחינה רעיונית, אך חסרה דיוק בניסוח.  הוכחת הקיום של N  צריכה להיות מפורשת יותר, ומסודרת יותר.  הניסוח 'איברים של (bk) שווים לאברים מתוך (an)' אינו מדויק מספיק.  רצוי להסביר מדוע  ההתכנסות של (an) גוררת התכנסות של (bk) בצורה פורמלית יותר. \\ \hline
2 & ב & 80 & הוכחה נכונה רעיונית, אך חסר פירוט.  יש להסביר בצורה ברורה יותר את הקשר בין N ו-N'.  הטענה 'k+m > N′ אז k > N'  דורשת נימוק נוסף.  הניסוח יכול להיות ברור יותר וקומפקטי יותר. \\ \hline
3 & א & 60 & הפתרון מראה ניסיון להוכיח את הגבול, אך הנימוק לקביעת N אינו נכון.  הטענה  'n < 1' אינה עקבית עם הדרישה n > N.  הפתרון צריך להראות כיצד נבחר N עבור ε נתון, כך ש- |an - 0| < ε עבור כל n > N.  הצבת N=1/ε+1 נראית שרירותית  ומבלי נימוק. \\ \hline
3 & ב & 40 & הנימוק אינו נכון.  האי-שוויון |bn - 0| ≤ 5+n/(4n\^{}2+1) < 5+n/n\^{}2 אינו נכון באופן כללי.  הפתרון לא מציג  שיטה לבחירת N  עבור ε נתון.  אין הסבר כיצד הגיע למסקנה שגבול הסדרה הוא 0. \\ \hline
3 & ג & 50 & הפתרון מראה ניסיון להוכיח את הגבול, אך הנימוק אינו מדויק.  השימוש באי-שוויון  √(16n\^{}2+3) < √(16n\^{}2+8n+1) אינו מספיק להוכחת הגבול.  יש להראות כיצד נבחר N עבור ε נתון, כך ש- |an - 4| < ε עבור כל n > N.  הצבת N=1/ε+1 נראית שרירותית  ומבלי נימוק. \\ \hline
3 & ד & 40 & הנימוק אינו נכון.  האי-שוויון √(n\^{}2+1) - n < 1/n אינו נכון באופן כללי.  הפתרון לא מציג  שיטה לבחירת N  עבור ε נתון. אין הסבר כיצד הגיע למסקנה שגבול הסדרה הוא 0. \\ \hline
3 & ה & 0 & הפתרון שגוי.  הוא אינו מראה  כי אין גבול,  והנימוק אינו ברור.  הטענה  '|en-L| > ε' אינה מוכחת בצורה מספקת. \\ \hline
3 & ו & 0 & הפתרון אינו ברור, ומבולגן.  אין הוכחה מסודרת  להתבדרות הסדרה. \\ \hline
4 & א & 95 & הוכחה נכונה וברורה של שני הכיוונים.  השימוש בסימונים ברור ומסודר.  יכול היה להיות קצת יותר קומפקטי. \\ \hline
4 & ב & 75 & הדוגמה נגדית נכונה.  אולם, ההסבר אינו מספיק ברור.  יש להסביר מדוע  lim n→∞ |an| = 1  אינו סותר  lim n→∞ an ≠ 1. \\ \hline
4 & ג & 90 & הוכחה נכונה וברורה של שני הכיוונים.  השימוש בסימונים ברור ומסודר. \\ \hline
5 & i & 100 & תשובה נכונה. \\ \hline
5 & ii & 100 & תשובה נכונה. \\ \hline
5 & iii & 100 & תשובה נכונה. \\ \hline
5 & iv & 100 & תשובה נכונה. \\ \hline
5 & v & 100 & תשובה נכונה. \\ \hline
5 & vi & 100 & תשובה נכונה. \\ \hline
5 & vii & 100 & תשובה נכונה. \\ \hline
\end{tabular}
\end{RTL}
\end{document}