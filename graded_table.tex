\documentclass{article}
\usepackage[utf8]{inputenc}
\usepackage{bidi}
\usepackage{geometry}
\geometry{margin=2.5cm}
\begin{document}
\section*{טבלת ציונים}
\begin{RTL}
\begin{tabular}{|c|c|c|p{10cm}|}
\hline
שאלה & סעיף & ציון & הערה \\
\hline
1 & 1.1 & 90 & הוכחה באינדוקציה נכונה מבחינה לוגית.  הניסוח יכול להיות משופר.  מומלץ להשתמש בסמלים מתמטיים מדויקים יותר (לדוגמה, שימוש ב- \\u2208 במקום '∈' ,  =def במקום 'def|\{z\}',  \\u2217 במקום '*' להכפלה) ולקצר את הניסוח.  השתדלות נאותה בניסוח, אך ניתן לשפר \\ \hline
1 & 1.2 & 80 & הוכחה באינדוקציה נכונה מבחינה לוגית, אך ישנם פערים בניסוח.  הבסיס לא מפורט מספיק.  המעבר מ-(xm)\^{}n * (xm) ל- x\^{}(mn) * x\^{}m אינו מנומק מספיק (אפשר להוסיף שלב ביניים עם חוקי חזקות).  הניסוח דורש שיפור ניכר. דגש על ניסוח מתמטי תקין \\ \hline
1 & 1.3 & 70 & ההוכחה לא ברורה מספיק.  המעבר מ- (x/y)\^{}n ל- x\^{}n * (1/y)\^{}n אינו מנומק מספיק. יש צורך להסביר את השימוש בחוקי חזקות בצורה ברורה יותר.  השימוש ב- 'lecture' אינו מתאים.  יש צורך בניסוח מתמטי מדויק יותר. \\ \hline
2 & None & 60 & הוכחה באינדוקציה לא שלמה.  חסר נימוק מפורט במעבר בין שלבים.  ההוכחה אינה מסבירה מספיק מדוע x < y גורר x\^{}m < y\^{}m  לכל m  (יש לשקול הוכחה על ידי אינדוקציה או  דרך אלטרנטיבית  עם נימוק מתמטי קפדני יותר).  השימוש ב- '⇐⇒' צריך להיות מוצדק  בצורה מתמטית יותר מדויקת. \\ \hline
3 & None & 50 & ההוכחה באמצעות סתירה לא ברורה דיה.  הנימוק למה  β+ε1>β  אינו מספיק.  ההסבר לקיומו של ε1  חסר.  המעבר מ- β+ε1 ≤ a לכל a∈A ל- \{β+ε1\} ≤ A אינו ברור.  יש צורך לנסח את ההוכחה בצורה ברורה ומדויקת יותר עם נימוקים מתמטיים. \\ \hline
4 & None & 75 & הוכחת שני הכיוונים אינה ברורה מספיק.  הטענה b < b+ε היא טריוויאלית ואינה מספקת נימוק להוכחה.   יש צורך בנימוקים מתמטיים מפורטים יותר.  הניסוח יכול להשתפר.  הוכחת שני הכיוונים עקרונית נכונה אך מנומקת בצורה לקויה \\ \hline
5 & 5.1 & 40 & הוכחה באמצעות סתירה לא שלמה.  הטענה (a1 > a2) ∧ (a2 > a1) אינה מנומקת  מספיק.  יש להסביר מדוע זהו סתירה להנחת היחידות.  הוכחת יחידות באמצעות סתירה דורשת נימוק מדויק יותר. \\ \hline
5 & 5.2 & 85 & הוכחה נכונה מבחינה לוגית, אך הניסוח יכול להיות מדויק יותר.  יש להסביר בצורה ברורה יותר את המעבר בין  ∃b∈R∀a∈A a < b ל- ∃b∈R∀a∈-A a > -b.  השימוש ב'נגדי' דורש הסבר נוסף.  הוכחה נכונה מבחינה לוגית \\ \hline
5 & 5.3 & 30 & הטענה A\\B ≥ B אינה מנומקת מספיק.  אין נימוק מספק מדוע A\\B אינו חסום מלמעלה.  הניסוח דורש שיפור ניכר.  הוכחה לא ברורה ולא מספקת.  חסר נימוק מתמטי. \\ \hline
6 & 6.1 & 20 & הנימוק לא חסום מלמעלה ומלרע אינו ברור.  ההוכחה חסרה נימוקים מתמטיים.  הניסוח לא ברור.  הפתרון לא מובן, חסר הסבר מתמטי תקין. \\ \hline
6 & 6.2 & 10 & הטענה אינה מנומקת מספיק.  אין נימוק מתמטי להוכחת חסום מלרע.  הדוגמה  ללא חסום מלמעלה אינה ברורה מספיק.  הפתרון אינו מובן, חסר הסבר מתמטי תקין.  חסר נימוק מתמטי \\ \hline
7 & 7.1 & 95 & הוכחה נכונה ומסודרת היטב. ניסוח ברור ומדויק.  הוכחה אלגברית נכונה \\ \hline
7 & 7.2.1 & 80 & הנימוק אינו מספיק. יש לנמק מדוע x1x3 + x2 ≥ 2.  ההוכחה לא מלאה \\ \hline
7 & 7.2.2 & 70 & ההוכחה באמצעות סתירה אינה מספקת.  יש צורך בנימוקים מתמטיים מדויקים יותר.  הניסוח לא מדויק \\ \hline
7 & 7.2.3 & 90 & הוכחה נכונה ומסודרת.  שימוש נכון בסעיפים הקודמים.  הוכחה נכונה ומסודרת \\ \hline
7 & 7.2.4 & 90 & ההוכחה טובה אך יש צורך להבהיר את טיעון הסימטריה בצורה ברורה יותר.  הניסוח יכול להיות מדויק יותר.  הוכחה נכונה וטובה \\ \hline
7 & 7.3 & 60 & ההוכחה באינדוקציה אינה מספיק ברורה.  המעבר מהנחת האינדוקציה למסקנה אינו מנומק דיו.  הניסוח דורש שיפור. \\ \hline
\end{tabular}
\end{RTL}
\end{document}